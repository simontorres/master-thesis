\NeedsTeXFormat{LaTeX2e}[2005/12/01]
%%    2011/08/29 Vorlage fuer Bachelor- und Masterarbeiten am Institut fuer Astrophysik Goettingen
%% Alle Aenderungen gegenueber der Vorlage von Thomas Pruschke (siehe unten) sind in CHANGELOG.txt im selben
%% Verzeichnis aufgelistet.
%%
%%    2009/03/12 v1.0 GAUBM Vorlage fuer Abschlussarbeiten Physik
%% Template fuer Bachelor- und Masterarbeiten
%% an der Fakultaet fuer Physik (c) Thomas Pruschke der GA Universit�t
%% Verbesserungsvorschlaege bitte an pruschke@theorie.physik.uni-goettingen.de
%%
%% Benoetigte Pakete: datenumber
%%

%%%%%%%%%%%%%%%%%%%%%%%%%%%%%%%%%%%%%%%%%%%%%%%%%%%%%%%%%%%%%%%%%%%%%%
%%%%%%%%%% Bitte vor dem Veraendern diese Datei umbenennen! %%%%%%%%%%
%%%%%%%%%%%%%%%%%%%%%%%%%%%%%%%%%%%%%%%%%%%%%%%%%%%%%%%%%%%%%%%%%%%%%%

%% scrbook - Ersatz f�r LaTeX book Klasse aus dem KOMA Script
%% Moegliche Optionen: diejenigen der Klasse scrbook ausser titlepage

%% deutsche Arbeit:
\documentclass[master,       %% Typ der Arbeit: bachelor oder master
               twoside,        %% zweiseitiges Layout
               BCOR10mm,       %% Bindekorrektur 10 mm
%               liststotoc,nomtotoc,bibtotoc, %% Aufnahme der div. Verzeichnisse
                                              %% ins Inhaltsverzeichnis
%               english,ngerman, %% Alternativspr. Englisch, Dokumentspr. Deutsch
               ngerman,english  %% Alternativspr. Deutsch, Dokumentspr. Englisch
%               final,          %% Endversion; draft fuer schnelles Kompilieren
               ]{GAUBM_astro}

\usepackage{aas_macros} %% Definiert Makros fuer Journale in Bibtex-Eintraegen von ADS

\usepackage{setspace}  %% Zur Setzung des Zeilenabstandes
\usepackage[ngerman]{babel}     %% Sprachen-Unterstuetzung
\usepackage{calc}      %% ermoeglicht Rechnen mit Laengen und Zaehlern
\usepackage[T1]{fontenc}       %% Unterstutzung von Umlauten etc.
\usepackage[latin1]{inputenc}  %% 
%% in aktuellem Linux & MacOS X wird standardmaessig UTF8 kodiert!
%\usepackage[utf8]{inputenc}    %% Wenn latin1 nicht geht ...

\usepackage{amsmath,amssymb} %% zusaetzliche Mathe-Symbole

\usepackage{lmodern} %% type1-taugliche CM-Schrift als Variante zur
                     %% "normalen" EC-Schrift
%% Paket fuer bibtex-Datenbanken
\usepackage[comma,sort&compress]{natbib}
\bibliographystyle{doktor_natb_bab_amp}

\newcommand{\tabheadfont}[1]{\textbf{#1}} %% Tabellenkopf in Fett
\usepackage{booktabs}                      %% Befehle fuer besseres Tabellenlayout
\usepackage{longtable}                     %% umbrechbare Tabellen
\usepackage{array}                         %% zusaetzliche Spaltenoptionen

%% umfangreiche Pakete fuer Symbole wie \micro, \ohm, \degree, \celsius etc.
\usepackage{textcomp,gensymb}

%\usepackage{SIunits} %% Korrektes Setzen von Einheiten
\usepackage{units}   %% Variante fuer Einheiten

%% Selbst hinzugefuegte Pakete %%%%%%%%%%%%%%%%%%%%%%%%%%%%%%%%%%%%%%%%%%%%%%%%%%%%%%%%%%%%%%%%%%%%%%%%%%%%%%
\usepackage{multirow}
\usepackage{lscape}
%\usepackage{ziffer}
%\usepackage{icomma}
%\usepackage{siunitx}
\usepackage[size=normalsize]{caption}
%\usepackage[clearempty]{titlesec}

%%%%%%%%%%%%%%%%%%%%%%%%%%%%%%%%%%%%%%%%%%%%%%%%%%%%%%%%%%%%%%%%%%%%%%%%%%%%%%%%%%%%%%%%%%%%%%%%%%%%%%%%%%%%%

%% Hyperlinks im Dokument; muss als eines der letzten Pakete geladen werden
\usepackage[pdfstartview=FitH,      % Oeffnen mit fit width
            breaklinks=true,        % Umbrueche in Links, nur bei pdflatex default
            bookmarksopen=true,     % aufgeklappte Bookmarks
            bookmarksnumbered=true  % Kapitelnummerierung in bookmarks
            ]{hyperref}

%% Weiter benoetigte Pakete: datenumber
%% Falls dieses Paket nicht in der Installation vorhanden ist,
%% kann es von der Seite mit diesem Template heruntergeladen werden
%% und in einem LaTeX bekanntem Verzeichnis installiert werden (notfalls
%% dem Verzeichnis mit der Arbeit).

%% Eigene Befehle %%%%%%%%%%%%%%%%%%%%%%%%%%%%%%%%%%%%%%%%%%%%%%%%%%%%%%%%%%%%%%%%%%%%%%%%%%%%%%%%%%%%%%%%%%%
% Hier koennen nuetzliche Abkuerzungen etc. definiert werden, z.B.:
\newcommand{\ci}{\textdegree}
\newcommand{\de}{\partial}
\newcommand{\dd}{\mathrm{d}}
\newcommand{\const}{\mathrm{const.}}

%%%%%%%%%%%%%%%%%%%%%%%%%%%%%%%%%%%%%%%%%%%%%%%%%%%%%%%%%%%%%%%%%%%%%%%%%%%%%%%%%%%%%%%%%%%%%%%%%%%%%

\begin{document}
%%
%%                   Ab hier muessen die Anpassungen geschehen
%%
%% Hier den eigenen Namen einsetzen
\ThesisAuthor{Sim\'on Neftal\'i}{Torres Robledo}
%% Hier den Geburtsort einsetzen
\PlaceOfBirth{Quilitapia, Chile.}
%% Titel Arbeit. Das erste Argument ist der deutsche, das zweite der
%% englische Titel.
\ThesisTitle{Einfluss der Umgebungstemperatur auf die Wellenl\"angen stabilit\"at von $I_2$-Gaszelle}{Influence of 
Environmental Thermal Stability on the Wavelength Solution of Iodine Cells.}
%% Erst- und Zweitgutacher/in
%% Ist der/die Betreuer/in nicht identisch mit dem/r Erstgutachter/in,
%% muss diese/r als optionales Argument angegeben werden.
\FirstReferee[Prof. Dr.\ Ansgar Reiners, Dr.\ Philipp Huke]{Prof. Dr.\ Ansgar Reiners}
\Institute{Institut f\"ur Astrophysik}
\SecondReferee{Dr.\ Philipp Huke}
%% Beginn und Ende des Anfertigungszeitraumes
\ThesisBegin{1}{3}{2015}
\ThesisEnd{30}{8}{2015}
%% DO NOT TOUCH THESE LINES!!!!
\frontmatter
\maketitle
\cleardoublepage
%% Zusammenfassung. Falls nicht gewuenscht, bitte auskommentieren.

%% So laesst sich in die andere Sprache umschalten (Englisch bzw. Deutsch)
\begin{otherlanguage}{english}
\begin{abstract}
  One of the most important aspects of data analysis for astrophysics is the calibration. One technique of wavelength 
calibration for spectroscopy is using atomic standard, for instance, Molecular Iodine ($I_2$). A \emph{Gas Cell} is a 
container usually made out of glass/pyrex in the shape of a cylinder with its interior filled with a gas. Molecular 
Iodine needs to be heated in order to be in gaseous state, from this temperature will depend the line depth (DELETE:: 
mainly and we'll se what else :: DELETE) therefore we worked on a way to have a very temperature stable Iodine$_2$ 
Cell. 
  We use a Fourier Transform Spectrograph (FTS) available at the Institut to perform 
  very high spectral resolution (R $\sim$ 600.000) measurements in order to characterize the \emph{spectral stability} in terms of line depth and wavelength
  solution. 
  \bigskip\par
  \textbf{Keywords:} Wavelength calibrators, Iodine Cells, Spectroscopy
\end{abstract}
\end{otherlanguage}

%% Ende des Vorspanns
\cleardoublepage
%% Ab hier 1 1/2 facher Zeilenabstand (durch setspace-Paket)
\onehalfspacing
%% Erzeugt Inhaltsverzeichnis
\tableofcontents

%% Hier kann man seine Bezeichnungsweisen erklaeren. Falls nicht
%% benoetigt, bis einschliesslich \end{nomenclature} auskommentieren
\begin{nomenclature}
%% Fuer die Berechnung der Spaltenbreiten muss \usepackage{calc}
%% geladen sein!
\section*{Lateinische Buchstaben}
\noindent
\begin{longtable}[l]{p{0.2\textwidth}p{0.7\textwidth-6\tabcolsep}p{0.1\textwidth}}
  \tabheadfont{Variable}&\tabheadfont{Bedeutung}&\tabheadfont{Einheit}\\\midrule\endhead
  $A$ & Querschnittsfl\"ache & $\unit{m^2}$\\
  $c$ & Geschwindigkeit & $\unitfrac{m}{s}$
\end{longtable}
\section*{Griechische Buchstaben}
\begin{longtable}[l]{p{0.2\textwidth}p{0.7\textwidth-6\tabcolsep}p{0.1\textwidth}}
  \tabheadfont{Variable}&\tabheadfont{Bedeutung}&\tabheadfont{Einheit}\\\midrule\endhead
  $\alpha$  & Winkel & $\unit{\degree}$; --\\
  $\varrho$ & Dichte & $\unitfrac{kg}{m^3}$
\end{longtable}
\section*{Indizes}
\begin{longtable}[l]{p{0.2\textwidth}p{0.8\textwidth-4\tabcolsep}}
  \tabheadfont{Index}&\tabheadfont{Bedeutung}\\\midrule\endhead
  R & Resolving Power\\
  SNR & Signal to Noise Ratio
\end{longtable}
\section*{Abk\"urzungen}
\begin{longtable}[l]{p{0.2\textwidth}p{0.8\textwidth-4\tabcolsep}}
  \tabheadfont{Abk\"urzung}&\tabheadfont{Bedeutung}\\\midrule\endhead
  FTS & Fourier Transform Spectrograph\\
  3D & dreidimensional\\
  max & maximal
\end{longtable}
\end{nomenclature}
%% \listoftables und \listoffigures sollten nur bei genuegender Anzahl Tabellen
%% verwendet werden
%\listoffigures
%\listoftables

\mainmatter   %% Anfang Hauptteil

\chapter{Introduction}
In the last decades there has been great progress in all the fields of science due to the also incredible advances in technologie.
In fact they are strictly correlated. Higher computational power, construction techniques, 3d printing, resources availability, new
materials

gas cells have been around a while ago (
\section{Motivation}
(scientific motivation)

\section{Concepts of Spectroscopy}
Hier sollte etwas Text stehen.
\subsection{Absorption and Emission Lines}
Noch ein paar Beispiele zu Abbildungen und Tabellen:
\subsection{Line Shapes}
Atomic spectra and stability
\section{Gas Cells}
Gas Cells are glass cilinder with optical windows on its extremes. The glass has to be of very low thermal 
expansion coefficient like \emph{borosilicate} (pyrex). (because is going to be glued to the optical windows therefore 
it would create mechanical tension, optical properties, reduces the risk of breaking)
\subsection{Iodine}
\subsection{Other Gases}
\subsection{Cell Heating}
\begin{center}
 \begin{figure}[h!]
 \centering
 \includegraphics[width=\columnwidth]{./img/diag_heatloss.png}
 % diag_hardware.png: 0x0 pixel, 300dpi, 0.00x0.00 cm, bb=
 \caption{Heat loss}
 \label{fig:diag_heatloss}
\end{figure}
\end{center}



\chapter{Experimental Design}

\section{Hardware}
\subsection{Optical Setup}
\subsection{TheBox}
\subsection{Monitoring and Control}
\subsubsection{Raspberry Pi}
\subsubsection{Arduino}
\subsubsection{Custom PCB}
\subsubsection{Thorlabs TC200}
\subsubsection{USB Switchbox}

\begin{center}
 \begin{figure}[h!]
 \centering
 \includegraphics[width=0.9\columnwidth]{./img/diag_hardware.png}
 % diag_hardware.png: 0x0 pixel, 300dpi, 0.00x0.00 cm, bb=
 \caption{Hardware description}
 \label{fig:diag_hardware}
\end{figure}
\end{center}



\section{Software}
\subsection{GNU/Linux}
\subsection{Python}
\subsection{Arduino}
\subsection{ZMQ}
\subsection{LAMP}
LAMP stands for: \emph{Linux, Apache, MySQL and PHP} is a widely common combinations of tools for setting up a standard 
web server
\begin{center}
 \begin{figure}[h!]
 \centering
 \includegraphics[width=0.9\columnwidth]{./img/diag_software.png}
 % diag_hardware.png: 0x0 pixel, 300dpi, 0.00x0.00 cm, bb=
 \caption{Software description}
 \label{fig:diag_software}
\end{figure}
\end{center}
\section{Instrument}
\subsection{Fourier Transform Spectrograph}


\chapter{Experimental Results}
\section{Data Acquisition}
\subsection{Dataset}
\subsubsection{FTS}
In order to fully characterize the gas cell behaviour with respect to temperature I performed the following 
measurements using the FTS
\begin{itemize}
 \item Stability test over 12 hours at several temperatures such as: Ambient temperature aprox 22, 30, 
40, 50, 60 FTS vacuum:
 \item Heating up from ambient temperature up to 60 degree
 \item Cooling down from 60 degree to ambient temperature
 \item Long term (10 hours) measure from lab temperature up to 60 degree (too see how long does it takes to stabilize 
line depth)
\end{itemize}
\begin{center}
\begin{tabular}{|c|c|c|c|c|c|}
\hline
Temp. & Stable & Duration & FTS Vacuum & FTS Pressure (hPa) & Date\\
\hline
30 & Yes & 5h & No & 1030.00 & 2015/08/21\\
\hline
30 & Yes & 5h & Yes & 0.262 & 2015/08/21\\
\hline
30 & Yes & 12h & Yes & 0.118 & 2015/08/22\\
\hline
40 & Yes & 12h & Yes & 0.0917 & 2015/08/24\\
\hline           
50 & Yes & 12h & Yes & 0.099 &2015/08/23 \\
\hline          
60 & Yes & 12h & Yes & 0.0917 & 2015/08/24\\
\hline
20 & No & 12h & Yes& 0.0901 &2015/08/25 \\
\hline
20-60 & Yes & 12h & Yes & 0.0779 & 2015/08/26\\
\hline              
60-20 & Yes & 12h & Yes & 0.0798 &2015/08/26 \\
\hline
%× & × & × & × & ×\\
%\hline
\end{tabular}
\end{center}


\section{Data Analysis}
In order to process the FTS data I developed Python tools, the key features in mind where:
\begin{itemize}
 \item Wavenumber range cut
 \item Spline Interpolation
 \item Two Spectra Cross Correlation
 \item Line Gaussian Fit
\end{itemize}

\section{Results}

\chapter{Future Work}



Text\dots
\chapter{Results}
Text\dots
\chapter{Discussion}


\appendix
\chapter{erster Anhang}
Text\dots
\chapter{zweiter Anhang}
Text\dots

\cleardoublepage
%% Bibliographie. Das Argument muss der Name der BIBTeX-Datenbank stehen.
%% Ein Beispiel fuer eine solche Datenbank finden Sie in bthesis_datenbank.bib
\bibliography{iodine} 

\chapter*{Danksagung}
Dank\dots

%% Dieser Befehl MUSS am Ende stehen und erzeugt die Erklaerung ueber die
%% benutzten Mittel
\Declaration
\end{document}
