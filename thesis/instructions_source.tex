\begin{center}
 INSTRUCTIONS TO SETUP RASPBIAN FOR THE \emph{THEBOX} EXPERIMENT

\end{center}


I will assume you ''are'' using linux as a development platform, therefore I will assume
as well that you know how to open a Terminal and how to enter commands. In this document I describe roughly the steps necesary to 
get TheBox working from scratch.


 \section{Download and Install Raspbian}
    \begin{enumerate}
    \item Download the latest raspbian disk image from:
    
         \verb=     https://www.raspberrypi.org/downloads/=
    \item Identify your SD card and Install Raspbian.\\
     Before plugging the SD card in the slot type in the terminal.
     
         \verb=     > ls /dev/sd*=\\
     Observe the output, it should look something like \emph{"/dev/sda"}. Say,
     
         \verb=     /dev/sda=\\
         \verb=     /dev/sda1=\\
         \verb=     /dev/sda2=\\
         %\emph{(/dev/sdXn)}\\
     In this example, \emph{/dev/sda} is the device and \emph{/dev/sda1} and \emph{/dev/sda2} are partitions.
     Then plug the SD card and repeat the previous command, the new elements in the output
     represent your SD card. The following is an example, change accordingly.
     
         \verb=     /dev/sdb=\\
         \verb=     /dev/sdb1=\\
         \verb=     /dev/sdb2=\\
     Make sure it is unmounted by typing.
     
         \verb=     > sudo umount /dev/sdb1=\\
         \verb=     > sudo umount /dev/sdb2=\\
    \item Now you can install Raspbian. From the folder where you downloaded the raspbian image.
    
         \verb+     > sudo dd if=./raspbian-image.img of=/dev/sdb bs=4M+\\
     if you have the raspbian on a different location just give the absolute path. \verb+bs=4M+
     may sometimes not work, if that's your case try with \verb+bs=1M+
     Wait until it is finished and then remove the SD card and plug it into your Raspberry Pi.
  \end{enumerate}
  \section{Power Up and initial Setup}
  
  Power Up the Raspberry Pi by plugging a proper Power Supply Unit.
  
  \subsection{With a Screen}
%   If you are using a screen  use only this \emph{first} instruction but in the Raspberry Pi. If you are connecting from another computer 
%     you are going to discover the RPi's IP address but first you need the IP address of
%     the \emph{ETHERNET} card in the computer that you are using for development. Plug an ethernet cable between the RPi and your computer for development. 
%     Go to a terminal and type the following command.\\
    Finding your computer's IP address is very straightforward. Just open a terminal and type:\\
    \verb=     > ifconfig=\\
    The output should give you a list of the network interfaces. Look for \verb=eth0= or something like that. If you don't feel comfortably about this
    do some research with google first until you have a better idea of what you are doing. Ignore \verb=lo= and \verb=wlan0=. 
    Take pen and paper and write down this, right after \verb=inet addr:= is the IP address, please be aware that the values 
    presented here are examples only.\\
    \verb=	inet addr:192.168.0.7=\\
    You can also do this in one single command:\\
    \verb=     > ifconfig eth0 | grep 'addr:' | awk '{print $2}' \=\\
    \verb=       | sed 's/addr://g'=\\    
    If you use the previous instruction just make sure that \verb=eth0= is the actual ethernet adapter that you are using for this connection. Following 
    this example, the output should be just:\\
    \verb=	192.168.0.7=\\
  \subsection{Remotely}
  \begin{enumerate}
   
   \item If you want to work remotely, which I recommend since less resources of the Pi will be used, discovering the Raspberry Pi's IP address can be 
   a bit more dificult.
    
    Now use  \verb=nmap= in order to scan though all the possible IP addressess and discover if any is active.\\
    \verb=     > nmap -sn 192.168.0.7/24=  \\
    Unfortunately at the moment of writing this document I don't have a Raspberry Pi at hand so I just used my laptop and my current ip is 
    \textbf{ xxx.xxx.97.221}. The output will be something like this. 
    \begin{verbatim}
Starting Nmap 6.40 ( http://nmap.org ) at 2015-12-09 18:25 CLT
Nmap scan report for xxx.xxx.97.0
Host is up (0.0021s latency).
Nmap scan report for xxx.xxx.97.1
Host is up (0.0016s latency).
.
.
.
Nmap scan report for xxx.xxx.97.248
Host is up (0.0013s latency).
Nmap scan report for csp2.lco.cl (xxx.xxx.97.251)
Host is up (0.00081s latency).
Nmap done: 256 IP addresses (32 hosts up) scanned in 3.23 seconds
\end{verbatim}
    If you don't get something similar try again and if still nothing do some research on the Internet.
    Most likely you will get less entries as output. 
    
    
    When I was looking for the RPi's IP it turned out to be:\\
    \verb=     10.42.0.91=\\
    It will be easy to tell once you have the output of nmap.
    
   
    \item Next step is connect through \verb=SSH= to the RPi. The latest \emph{Raspbian} comes with SSH activated by default.\\
    \verb=     > ssh pi@10.42.0.91=\\
    The default password is \verb=raspberry=.
   \item From now on most of the commands will be on a RPi's terminal, either from a screen or from the development computer through SSH.
    If you are using a screen (and keyboard) connected to the RPi \verb=raspi-config= will pop up automatically during boot, if not type:\\
	  \verb=     > sudo raspi-config=\\

    \begin{figure}[h!]
 \centering
 \includegraphics[width=0.7\textwidth,bb=0 0 724 486]{./img/a1-raspi-config-0.png}
 % a1-raspi-config-0.png: 724x486 pixel, 72dpi, 25.54x17.15 cm, bb=0 0 724 486
 \caption{This is the first raspi-config screen, from here we will use option 1,2,5 and 9.}
 \label{fig:a1-raspi-config}
\end{figure}

%    \subsection{With a Screen}\label{sect:withscreen}
    
    \item Inital Setup with raspi-config will requiere some data which you should have at hand. Is important that you first check
    the RPi's specific documentation. Right now you will do the following steps in raspi-config
    \begin{figure}[h!]
 \centering
 \includegraphics[width=0.7\textwidth,bb=0 0 724 486]{./img/a1-raspi-config-advanced.png}
 % a1-raspi-config-advanced.png: 724x486 pixel, 72dpi, 25.54x17.15 cm, bb=0 0 724 486
 \caption{Select option 9 in the main menu to select one of the Advanced Options displayed here.}
 \label{fig:a1-raspi-config-advanced}
\end{figure}

    \begin{itemize}
     \item Expand filesystem\\ Simply hit enter in the first option and say yes to all.
     \item Set New password for user \verb=pi=\\ This is an important safety measure, option 2.
     \item Set timezone\\ Use option 5.
     \item Enable I2C\\ Use option A7, Fig. \ref{fig:a1-raspi-config-advanced}.
     
   
    \end{itemize}
    Later you will need to run raspi-config again in order to change the Hostname.
   \end{enumerate}
   
   

  \section{Advanced Setup}
  \subsection{Upgrade and Network Settings}
 \begin{enumerate}

  \item At this point we will update the repositories lists and upgrade the system to its latest stable version.\\
  \verb=     > sudo apt-get update=\\
  %\verb=     > sudo apt-get upgrade=\\
  \verb=     > sudo reboot=\\
  We do this step now because in the past this has failed so it is early enough to go back and start over, of course if it fails once I woud recommend
  not doing it next time. The current version should be already a stable one.
  \item Now the time is right to configure the network properties. I prefer to use \verb=vim= as text editor, so we will jump ahead a little bit 
  and install it. You can skip this and install or use any other text editor.\\
  \verb=     > sudo apt-get install vim=
  
  We will need to change the hostname first so we go again to \verb=raspi-config= and change the computer hostname to \verb=viscacha=.
  We don't do this the first time we run raspi-config because that causes problems when trying to login again if we haven't set the ip 
  address correctly. Remember that the information that you are about to introduce must be obtained from an authorized person of your institution.
  
  \begin{figure}[h!]
 \centering
 \includegraphics[width=0.7\textwidth,bb=0 0 724 486]{./img/a1-raspi-config-hostname.png}
 % a1-raspi-config-hostname.png: 724x486 pixel, 72dpi, 25.54x17.15 cm, bb=0 0 724 486
 \caption{hostname}
 \label{fig:a1-hostname}
\end{figure}

  
  Now edit \verb=/etc/network/interfaces=\\
  \verb=     > sudo vim /etc/network/interfaces=\\
  It should look like this:\\
  \begin{verbatim}
auto lo
iface lo inet loopback

auto eth0
allow-hotplug eth0
iface eth0 inet static

address 134.76.204.166
netmask 255.255.254.0
network 134.76.204.0
broadcast 134.76.255.255
gateway 134.76.204.254
  \end{verbatim}
  
  Beware that after this process you will not be able to access the RPi from your computer unless it is part of the internal Network of the 
  department. So is OK to leave this process to the end. Also the specific numbers might change so talk with the IT person.
  
  If you are confidant that the values are correct, reboot the pi.
\end{enumerate}

  
  \subsection{Manage Users}
  As a safety measure it is recommended to not use default values, since in that case you will be opening the gates to potential outsiders with not so good intentions.
 \begin{enumerate}
 \item Create user "thebox" \\%password 42tQb8Qa4rSHrX5
         \verb=     > sudo adduser thebox=\\
     Enter a good password when requested. Don't forget to write it somewhere.
  \item Add user to ``sudoers`` and to ``dialout'' groups.\\
         \verb=     > sudo adduser thebox sudo=\\
         \verb=     > sudo adduser thebox dialout=
  \item Allow to run \verb=sudo= commands without password (Optional). We need to edit \verb=/etc/sudoers=\\
         \verb=     > sudo visudo -f /etc/sudoers=\\
     At the end add the following line
     
         \verb+     thebox ALL=(ALL) NOPASSWD: ALL+\\
     Take advantage of the situation and delete the line corresponding to the \verb=pi= user.
     For some reason it is only possible to edit this file with the \verb=visudo= editor,
     in order to save it press \verb=Ctrl-O= and to exit \verb=Ctrl-X=. Check that the line was
     successfully added. Now you can login as \verb=thebox= and continue.
     
     \end{enumerate}
     
     \subsection{Configure SSH}
     \begin{enumerate}
   \item Create ssh public keys. You might need to do this twice, in your development computer and in your RPi.\\   
         \verb=     > ssh-keygen=\\
     My suggestion would be don't enter any passphrase because otherwise you will need to
     type in every time you want to access. Unless you need to work under the most strict security settings. No-passphrase is safe enough for now.
   See image \ref{fig:a1-ssh-keygen} for an idea of what is should look like.
     
     \begin{figure}[h!]
 \centering
 \includegraphics[width=0.7\textwidth,keepaspectratio=true]{./img/a1-ssh-keygen.png}
 % a1-ssh-keygen.png: 697x486 pixel, 72dpi, 24.59x17.15 cm, bb=0 0 697 486
 \caption{Creation of public keys.}
 \label{fig:a1-ssh-keygen}
\end{figure}

    \item Add the RPi's \verb=id_rsa.pub= to your development's \verb=authorized_keys=. and vice versa, this allows to connect without typing a password from the owner of
    \verb=id_rsa.pub= that is being pasted into \verb=authorized_keys=. Execute this command both in the Raspberry Pi and in your development computer:
    
 	\verb=     > cat ~/.ssh/id_rsa.pub | ssh user@server \=\\
 	\verb=       "cat >> ~/.ssh/authorized_keys"=\\
     where \verb=user= and \verb=server= must be changed to the proper values. I'd suggest to do it the other way around as well. From your \verb=server= or development
     computer type this:\\
     \verb=     > cat ~/.ssh/id_rsa.pub | ssh thebox@viscacha "cat \=\\
     \verb=       "cat >> ~/.ssh/authorized_keys"=\\
     Hopefully this will be the last time you type your very safe password.
     \item A very useful tool is \verb=sshfs= which allows you to mount a remote directory in your computer using \verb=ssh=. This is particularly useful when developing code.
 \end{enumerate}
 
 \section{Software Installation}
  In this section I will explain what software you need to instal and how to do it. All the programs or tools in the following list
  are required in TheBox experiment. The instalation process is tested to work in this sequence but use your criteria to tell if 
  you can alter the order. If you just updated the repositories list in the previous step is not necesary to do it again, but if many days has passed do:\\
     \verb=     > sudo apt-get update=\\
  \begin{enumerate}
   \item Install python.\\
     \verb=     > sudo apt-get install python-dev python-numpy python-scipy \=\\
     \verb=       python-matplotlib ipython=\\
  \item Install apache2 and php5  \\
     \verb=     > sudo apt-get install apache2 php5 libapache2-mod-php5=\\
  Restart apache service with:\\
     \verb=     > sudo service apache2 restart=\\
  \item Install mysql.\\
     \verb=     > sudo apt-get install mysql-server mysql-client php5-mysql \=\\
     \verb=       python-mysqldb python-mysql.connector=\\
  A password for the root user will be requested, so have it prepared or put one and write it down.
  
  If you want to edit advanced features you need to edit \verb=/etc/mysql/my.cnf=.
  
  \subsection{Configure MySQL}
  This step is very important and creates users and provide grants for them in mysql. Note that this users are not \emph{system's users} they 
  will work in MySQL's context only.
  
  \verb=     > mysql -u root -p<add password here>=\\
  Now you will get a different prompt which means that you are no longer in the operating system environment but in mysql's. So, get familiar with
  its sintaxis prior to move on.
  \begin{figure}[h!]
 \centering
 \includegraphics[width=0.7\textwidth,keepaspectratio=true]{./img/a1-mysql.png}
 % a1-mysql.png: 724x486 pixel, 72dpi, 25.54x17.15 cm, bb=0 0 724 486
 \caption{This is what you should see.}
 \label{fig:a1-mysql}
\end{figure}
\begin{enumerate}
 \item First thing to do is to create two users.
  \begin{verbatim}
CREATE USER 'thebox'@'localhost' IDENTIFIED BY '<add password here>';
CREATE USER 'vultur'@'localhost' IDENTIFIED BY '<add password here>';
\end{verbatim}
Put different passwords and take note of them.

The \emph{grant} works similar to permissions in an operating system. The user \verb=thebox= will be able to do everything on the tables but is meant to
work only from the inside of the system. On the other hand, the user \verb=vultur= will be able only to read the tables and that user can be used from
the outside, i.e. web service.
\begin{verbatim}
GRANT ALL ON *.* TO 'thebox'@'localhost';
GRANT SELECT ON *.* TO 'vultur'@'localhost';
\end{verbatim}



\item Create a database so you can place the tables.\\ \verb=CREATE DATABASE theBoxData;=\\
And then \verb=exit= MySQL.

\item If you are starting a completely new project you will not have a backup from where to restore the database, but in this case you do have one. 
Make sure to obtain the right database backup, in this case is called \verb=theboxdata.sql=.
Now import the database in the backup files to mysql.\\
\verb=     > mysql -u root -p theBoxData < theboxdata.sql=

Note that \verb=theboxdata.sql= is the name of this particular backup and it could change.


If you don't have a backup you will need to create the tables by yourself. Login as mysql root user again and type this.

\verb=USE theBoxData;=\\
This is important because it tells MySQL where to place the new table.

Make sure to type all the following text, until the semicolon.
\begin{verbatim}
 CREATE TABLE temperatureAndStatusPhaseTwo ( 
 DATE TIMESTAMP NOT NULL, 
 SENSOR_ONE DECIMAL(6,2), 
 SENSOR_TWO DECIMAL(6,2), 
 SENSOR_THREE DECIMAL(6,2), 
 SENSOR_FOUR DECIMAL(6,2), 
 SENSOR_FIVE DECIMAL(6,2), 
 SENSOR_SIX DECIMAL(6,2), 
 SENSOR_SEVEN DECIMAL(6,2), 
 TH_SET_TEMP DECIMAL(6,2), 
 TH_VAL_TEMP DECIMAL(6,2), 
 TH_HEATER_ON BOOLEAN, 
 AIR_HEATER_ON BOOLEAN
 );
\end{verbatim}


Also, when running the temperature control system, some data is stored in another table. This data is used to keep track of the current
status, should a reboot or crash happen. It will resume with the same settings afterwards.

\begin{verbatim}
CREATE TABLE thermalControl (
TSET_TIME TIMESTAMP NOT NULL, 
TSET_VALUE DECIMAL(6,2), 
LAST_HEATER_STATUS BOOLEAN
);
\end{verbatim}   
\end{enumerate}

\item Several other tools are needed:
\begin{enumerate}
 \item ZMQ is a message broadcasting system (see \verb=http://zeromq.org/=)\\
\verb=sudo apt-get install libzmq-dev python-zmq=
 \item Python Serial is needed for reading serial ports using python. Very important.\\
\verb=sudo apt-get install python-serial=
\item INO is a command line Arduino tool. Allows you to upload code into the board using command line.\\
\verb=git clone git://github.com/amperka/ino.git=\\
\verb=cd ino=\\
\verb=sudo make install=
\item The Arduino that comes in the repositories is usually an old one. Doing this is usually better. But be aware that the newest version will
be different, so I recommend you to go to \verb=www.arduino.cc= and get the correct link.\\
\verb=wget http://downloads.arduino.cc/arduino-1.6.5-linux32.tar.xz=\\
\verb=tar xf arduino-1.6.5-linux32.tar.xz=
\item Other tools needed.\\
\verb=sudo apt-get install python-rpi.gpio=\\
\verb=sudo apt-get install python-setuptools=
\end{enumerate}

\item Compile USBswitch for Raspberry Pi. The SwitchBox is an USB controled switch, you plug it to a wall power outlet and then conect to it a
device that you want to control. In order to control you need to send a signal through an USB port, since the protocol for sending such instructions are
too complicated to developt one's own code it is easier to use their example but it comes pre-compiled for \verb=i386= architecture, therefore is necessary
to compile it for the \verb=ARM= arquitecture.\\
First install the necessary libraries:\\
\verb=     > sudo apt-get install libusb-dev=\\
\verb=     > sudo apt-get install libstdc++6=\\
\verb=     > sudo apt-get install libc6=\\
\verb=     > sudo apt-get install libgcc1=\\
Go to SwitchBox manufacturer's webpage  \\
\verb=http://www.antrax.de=\\
Search for the product's page of the SwitchBox device and click the \emph{Downloads} tab and then click on \emph{linux}.
Save the file and unpack it with:\\
\verb=     > mkdir antrax=\\
\verb=     > cd antrax=\\
\verb=     > unzip ../switchboxusb-linux.zip=\\
\verb=     > unzip AntraxAccess.zip=\\
\verb=     > unzip AntraxLinux.zip=\\

You will find a \emph{pdf} document with instructions in German: \verb=Switchbox-USB-Linux.pdf=\\
%This comes precompiled for \verb=i386=. We need to compile it for \verb=ARM=.\\
In order to proceed to the compilation, do the following:\\
\verb=     > cd USBaccess=\\
\verb=     > vim USBaccessBasic.c=\\
Change the line \verb=14= from \verb=#include <linux/usb.h>= to \verb=#include <usb.h>=\\
\verb=     > cc -c -lusb -o USBaccessBasic.o USBaccessBasic.c=\\
\verb=     > rm USBaccess.o=\\
\verb=     > make=\\
\verb=     > cp USBaccess.a USBaccess.h USBaccess.o ../Example/=\\
\verb=     > cd ../Example=\\
\verb=     > rm USBswitch USBswitch.o=\\
\verb=     > make=\\
At this point the command-line tool is ready to work but some work is still needed.\\
\verb=     > ls /dev/usb/=\\
The output should show something like this:\\
\verb=       hiddev0 hiddev1=\\
If nothing comes out is likely that \verb=udevd= is not working. It's possible to find out by typing:\\
\verb=     > pidof udevd=\\
Create a new rule. I think the best way proceed is as follows:\\
\verb#     > sudo echo 'BUS=="usb", KERNEL=="hiddev[0-9]", NAME="usb/hiddev%n",\#\\ 
\verb# MODE="0666"' >> /etc/udev/rules.d/10-local.rules#\\
If it doesn't work type \verb=sudo su= and you will be root. Now you can run the previous command.
Restart the service:\\
\verb=     > sudo /etc/init.d/udev restart=\\
\verb=     > cd USBaccess=\\
\verb=     > source make_hid=\\
By default we need to run \verb=USBswitch= as \verb=root= or with \verb=sudo=, this is just a matter of permissions but is not a good
idea just to change the permissions of the device. Do the following:\\
\verb=     > sudo chmod g+rw /dev/usb/hiddev0=\\
\verb=     > sudo  chgrp dialout /dev/usb/hiddev0=\\
Now you can turn on or off the SwitchBox.\\
\verb=     > USBswitch 1= (on)\\
\verb=     > USBswitch 0= (off)\\
After a reboot the permissions of the file \verb=/dev/usb/hiddev0= will return to normal so you might need to do this everytime. A script exists in the
latest system called \verb=setUSBswitch=.

\section{Adding a Real Time Clock (RTC)}
Since the Raspberry Pi is meant to be a cheap computer it doesn't come with an internal clock, so its recomended to use an RTC.
There is one implemented that uses the I2C protocol. \\
\verb=sudo apt-get install python-smbus=\\
\verb=sudo apt-get install i2c-tools=

We will need adafruit's instructions to setup correctly the RTC.\\
\verb=https://learn.adafruit.com/adding-a-real-time\=\\
\verb=-clock-to-raspberry-pi/overview=\\
The following instructions are reproduced from there, I will asume you have a revision 2 Raspberry Pi. At this point the RTC should be already pluged correctly.
In order to verify that, type:

\verb=     > sudo i2cdetect -y 1=

\begin{figure}
 \centering
 \includegraphics[width=0.7\textwidth]{./img/a1-i2cdetect.png}
 % i2cdetect.png: 472x252 pixel, 72dpi, 16.65x8.89 cm, bb=0 0 472 252
 \caption{This is what you should see. If instead of 68 you get UU, it means that the RTC has already been setup.}
 \label{fig:i2cdetect}
\end{figure}

If everything is working fine, move to the next level. First load the RTC kernel module:\\
\verb=     > sudo  modprobe rtc-ds1307=\\
Then you will need to be root, in raspbian (or any debian distro) the following instruction makes you root.\\
\verb=     > sudo su=\\
\verb=     > echo ds1307 0x68 > /sys/class/i2c-adapter/i2c-1/new_device=\\
Type \verb=exit= to go back to your normal user.

Check the time of the RTC with \verb=sudo hwclock -r= and if it is the first time this has been done you should get the 1st of January of 2000.

Check the time of your Raspberry Pi with \verb=date= If it is not correct run \verb=sudo raspi-config= and set the right timezone, if you are conected
to the internet that should fix the time properly. Now check the time again with \verb=date= and then write it to the RTC.

\verb=     > sudo hwclock -w= 

Again, you can check it with:

\verb=     > sudo hwclock -r=

If everything is correct at this point it means you have a running RTC, but at the next reboot it will not be working so we need to make the RTC kernel
module to load at boot time.

\verb=     > sudo vim /etc/modules=\\
At the end of the file add \verb=rtc-ds1307=

Now for the creation of the RTC device at boot time edit \verb=/etc/rc.local=\\

\verb=     > sudo vim /etc/rc.local=

Add, \emph{BEFORE} \verb=exit 0=.\\
\verb=echo ds1307 0x68 > /sys/class/i2c-adapter/i2c-1/new_device=\\
\verb=sudo hwclock -s=

\begin{figure}[h!]
 \centering
 \includegraphics[width=0.7\textwidth,keepaspectratio=true]{./img/a1-rc-local.png}
 % rc-local.png: 634x486 pixel, 72dpi, 22.37x17.15 cm, bb=0 0 634 486
 \caption{This is what the file \emph{/etc/rc.local} should look like.}
 \label{fig:rc-local}
\end{figure}

\end{enumerate}

\section{Webpage} [this part on neeeds some work]
Place the webfiles in \verb=/var/www/http/=
\section{TheBox Control Software}
\verb=     > mkdir ~/.bin=\\
Move the files there\\
edit \verb=.bashrc=\\
\verb=     > vim .bashrc=\\
add the line\\
\verb+export PATH=$PATH:/home/thebox/.bin+
\section{Clean Up}
bllbala